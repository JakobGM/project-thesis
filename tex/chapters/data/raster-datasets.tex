\begin{figure}[htb]
  \includegraphics[width=0.75\linewidth]{data/rgb-example}
\end{figure}

\begin{figure}[htb]
  \includegraphics[width=0.75\linewidth]{data/lidar-example}
\end{figure}

\begin{table}[htb]
  \centering
  \resizebox{\textwidth}{!}{%
    \begin{tabular}{lSSrrrr}
      \toprule
      {Data set} & {Resolution} & {Size} & {Coverage area} & {Date} & Scan angle & Accuracy \\
      \midrule
      Orthophoto~\cite{trondheim_ortophoto_2017} & \SIrange{0.04}{0.15}{\meter} & \SI{161}{\giga\byte}  & TODO & \date{2017} & {} & \SI{\pm 0.35}{\meter} \\
      LiDAR~\cite{trondheim_lidar_2017} & \SI{0.2}{\meter} & \SI{25}{\giga\byte} & \SI{342}{\kilo\meter\squared} & \date{2017-10-10} & \SI{\pm 20}{\degree} & \SI{\pm 0.02}{\meter} (SD) \\
      \bottomrule
    \end{tabular}%
  }
\end{table}

\begin{table}[htb]
  \centering
  \begin{tabular}{cc}
    \toprule
    {Point density (\si{\per\meter\squared})} & {Proportion (\%)} \\
    \midrule
    $> 100\%$ & 97.7 \\
    \SIrange{85}{100}{\percent} & 1.2 \\
    \SIrange{60}{85}{\percent} & 1.1 \\
    \bottomrule
  \end{tabular}
  \caption{
    Control of point cloud density of the Trondheim 2017 LiDAR data set.
    The densities are calculated within rolling windows of size $\SI{10}{\meter} \times \SI{10}{\meter}$~\cite{trondheim_lidar_2017}.
    }
\end{table}

Pixel domain is $(1, 254)$ for all three color channels.
\SI{70.77}{\percent} of all pixels are valid, probably due to lower resolution than the actual resolution.
Elevation data is in domain $(-9.390, 569.050)$.
