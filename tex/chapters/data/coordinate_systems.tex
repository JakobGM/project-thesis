\begin{sectionsummary}
  Explain the concept of coordinate systems and projections in context of geographical datasets.
  Explain how UTM zones are easier to work with, and how code that works with UTM coordinates
  are generalizable to all regions on earth.
\end{sectionsummary}

When it comes to geographical data, as with most things, different \textit{coordinate systems} have different trade-offs.
Coordinate system \textit{projections} are used in order to map a given set of coordinates to another coordinate system.
The target coordinate system might have a domain which is a proper subset of the source coordinate system, and the target system is not necessarily of the same dimensionality.
Cartographers must apply a projection in order to represent the three-dimensional spherical shape of the earth on a two-dimensional surface.
For instance, the common \textit{Mercator map projection} is suitable for navigation since a path of straight bearing is represented as a straight line on the resulting map, but the projection does \textit{not} preserve area as surfaces towards the poles become elongated \cite[p.~38]{map_projections_1987}.

A spherical coordinate system is most suitable for representing \textit{arbitrary} positions on the earths surface.
The most common coordinate system is the \textit{geographic coordinate system} (GPS).
A given point is uniquely represented by three scalars, $\vec{p} = (\phi, \lambda, z)$.
The latitude, $\phi$, is the angle between the equitorial plane and the line connecting the point to the center of the earth.
Likewise, the longitude, $\lambda$, is the angle between the same line and the reference meridian passing through Greenwich, England.
The elevation, $z$, is the radial distance from sea level to the given point.
Negative values for $z$ do not necessarily imply that the given point is below the ground, as certain areas (such as in the Netherlands) are situated below sea level.
It is therefore not sufficient to represent elevation data with unsigned floating point numbers.

Even though GPS is able to uniquely represent geographic positions with a high degree of accuracy, it is unsuitable for many applications.
For instance, cartesian transformations and norms are cumbersome to calculate, and data structures and visualizations which are fundamentally two dimensional, such as maps, rasters, and matrices, become difficult to use with a spherical coordinate system.

In order to solve this problem we define a set of coordinate system \textit{projections} which approximates given regions of the earth surface as being flat planes.
The resulting coordinate system is cartesian, and thus allows you to represent geographic points in the more common $\vec{p} = (x, y, z)$ format.
Cartesian distance norms such as $||\vec{p}_1 - \vec{p}_2||_2$ and cartesian translations $\vec{p}_1 + \vec{\Delta}$ stay within pre-defined error tolerances as long as operations are contained to the given validity region of the given projection.

One such cartesian approximation of the earth's surface is the Universal Transverse Mercator (UTM) coordinate system which divides the earth into 60 rectangular zones. The UTM zones covering Europe are shown in \figref{fig:europe-utm-zones}.

\begin{figure}[H]
  \centering
  \includegraphics[width=0.6\linewidth]{europe-utm-zones.png}
  \caption{
    The figure shows the UTM zones required in order to cover the entirety of Europe, from \texttt{29S} to \texttt{38W}.
    This public domain image has been sourced from Wikimedia \cite{wiki:europe_utm_zones}.
  }
  \label{fig:europe-utm-zones}
\end{figure}

For our dataset situated in Trondheim, Norway, we will therefore use UTM zone \texttt{32V}.
Data in other coordinate systems will be transformed to this UTM zone.
Since this is an affine coordinate system, we can easily generalize any models to other coordinate systems by applying the correct affine transformation.
