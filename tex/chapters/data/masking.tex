In order to create a ground truth segmentation mask we must convert the vector-formatted mask polygons, building outlines in our case, into the same rasterized format as the remote sensing data.
The construction of discretized segmentation masks from vectorized mask polygons is performed by~\algref{alg:masking}.

\begin{algorithm}{Discretized masking}{alg:masking}
  \item Transform the mask polygons into the pixel coordinate system of the raster tile, using the affine transformation defined by the tile origin.
  \item Superimpose the polygon on the discretized pixel grid and crop polygons outside the pixel region $(0, 255) \times (0, 255)$.
  \item Fill in the value $1$ for any pixel contained by the polygon exterior hulls, while not contained by any interior hull.
  \item Set remaining values to $0$.
\end{algorithm}

A problem arises when pixels are partially contained by a polygon exterior and interior, i.e.\ when the pixel overlaps the polygon's boundary.
The pixel must be rather arbitrarily considered as either contained (decision rule A) or not contained (decision rule B) by the polygon.
Both decision rules are shown in~\figref{fig:pixel-containment}.

\begin{figure}[H]
  \centering
  \begin{tikzpicture}[every node/.style={minimum size=0.6*1.3cm-\pgflinewidth, outer sep=0pt, fill=orange!80, fill opacity=0.4}, scale=0.6*1.3]
  \tikzmath{
    \gridheight=4;
    \gridwidth=4;
  }

  \def\cells{
    (0.5,0.5),
    (1.5,1.5),
    (2.5,2.5),
    (3.5,3.5),
    (3.5,2.5),
    (3.5,1.5),
    (3.5,0.5),
    (2.5,0.5),
    (1.5,0.5),
    (2.5,1.5),
    (2.5,3.5),
    (1.5,2.5)
  };
  \coordinate (offset) at (0.5, 0.5);
  \foreach \cell in \cells {
    \node at ($(\cell$) {};
  }

  \draw[step=1,color=black,draw opacity=0.6] (0,0) grid (\gridwidth,\gridheight);
  \draw[thick, orange, line width=0.1cm] (0.5, 0)
    -- (3.33, \gridheight)
    -- (\gridwidth, \gridheight)
    -- (\gridwidth, 0)
    -- cycle;
\end{tikzpicture}

  \hspace{2em}
  \input{tikz/non_touching_mask.tex}
  \caption{%
    The same polygon discretized to a raster grid using two different techniques.
    In the left figure, all pixels being \textit{touched} by the interior of the polygon
    are considered a part of the polygon (decision rule A), while in the left figure, only pixels
    entirely \textit{contained} within the interior are considered being part
    of the polygon (decision rule B).
  }%
  \label{fig:pixel-containment}
\end{figure}

An alternative is to average the two masks, resulting in mask values of $0.5$ where the two decision rules disagree.
Approximately \SI{9.2}{\percent} of mask pixels of value $1$ are situated along the boundary of a discretized mask polygon (\SI{1.7}{\percent} of \textit{all} pixels regardless of value) and may therefore be affected by this decision.
We have opted for decision rule A.
The distribution of the mask class balance across all produced tiles is shown in~\figref{fig:mask-class-balance}.

\begin{figure}[H]
  \centering
  \includegraphics{mask-balance}
  \caption{%
    Distribution of \textit{building density} across all produced tiles in Trondheim.
    Building density is defined by number of pixels positioned on top of buildings divided by total number of pixels.
  }%
  \label{fig:mask-class-balance}
\end{figure}

The average tile has a building density of approximately \SI{17}{\percent}, that is \SI{700}{\meter\squared} of \SI{4096}{\meter\squared} is occupied by buildings.
Of all the produced tiles approximately \SI{8.32}{\percent} end up having no positive mask pixels, i.e.\ no buildings are situated within these tiles.
