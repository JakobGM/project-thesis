Geographic data is in wide use by both the public and private sector, and is a huge subject in of itself.
Standards and best practices have emerged over time as a result.
The storage, processing, and inspection of such data is handled by so-called \textit{Geographic Information Systems} (GIS), and there exists educational programs and job descriptions specifically for "GIS Engineering".
There is a lot of specific concepts and nomenclature in use in the field, and it can feel quite daunting at first glance.
We will therefore start by explaining a few core GIS concepts relevant for the problem at hand, concepts which will inform decisions for how to prepare the data for our purposes.
