Raster data is simply scalar measurements imposed onto a grid with rectangular grid cells.

A color image, $I$, of width $\mathrm{W}$ and height $\mathrm{H}$, for instance, will contain three color channels,
\textcolor{red}{red}, \textcolor{green}{green}, and \textcolor{blue}{blue}, (RGB)
and can be represented by a three-dimensional array of size $\mathrm{W} \times \mathrm{H} \times \mathrm{3}$.
Each color channel in a given pixel is represented by an unsigned 8-bit integer, i.e.
\begin{equation*}
  I_{i, j, c} \in \{0, 1, \ldots, 255\},
  \hspace{2.5em}
  i = 0, \ldots, \mathrm{H} - 1,
  ~~
  j = 0, \ldots, \mathrm{W} - 1,
  ~~
  c = \mathrm{\textcolor{red}{R}, \textcolor{green}{G},\textcolor{blue}{B}}.
\end{equation*}

A LiDAR height map, $Z$, is likewise encoded as a single-channel grayscale image of size $\mathrm{W} \times \mathrm{H}$.
Each pixel is represented by a signed 32-bit floating point value, i.e.
\begin{equation*}
  Z_{i, j} \in \mathbb{R},
  \hspace{2.5em}
  i = 0, \ldots, \mathrm{H} - 1,
  ~~
  j = 0, \ldots, \mathrm{W} - 1.
\end{equation*}

These two raster types must be handled differently during data normalization due to their different value domains, which we will come back to in~\secref{sec:raster-normalization}.

% Keep the following section on the same page
\begin{minipage}{\textwidth}
  For GIS rasters we must define the spatial extent of the given image as well, specified by:
  \begin{itemize}[noitemsep]
    \item A coordinate system, for example UTM \texttt{32V}.
    \item The coordinate of the center of the upper left pixel, $I_{1, 1}$; the \textit{origin} $\vec{r}_0 = {[x_0, y_0]}^T$.
    \item The pixel step size, $\vec{\Delta} = {[\Delta_x, \Delta_y]}^T$, for example ${[\SI{0.25}{\meter}, \SI{-0.25}{\meter}]}^T$.
  \end{itemize}
\end{minipage}
\vspace{0.5em}

The pixel $I_{i, j, c}$ therefore represents a rectangle of width $\Delta_x$ and height $\Delta_y$ centered at the spatial coordinate $\vec{r}_0 + \vec{\Delta} \cdot [i, j]$, everything being interpreted in the given coordinate system.

Missing data in remote sensing rasters are specified by filling in a pre-defined \texttt{nodata} placeholder value.
For RGB data this is often set to $0$, resulting in a black pixel.
LiDAR rasters often use $\texttt{nodata} = -2^{127} \times (2 - 2^{-23}) \approx 3.4028234664 \times 10^{38}$, the smallest normal number representable by a single-precision floating point number.
Such \texttt{nodata} values may arise from measurement errors and regions which are outside the given covarage area of the dataset, and must be special-cased during data normalization, which we will come back to in~\secref{sec:raster-normalization}.

\subsubsection*{Merging raster datasets}

Aerial photography and LiDAR data is usually provided in several smaller raster files, organized in a tiled pattern in order to reduce file sizes.
Each file is a \texttt{.geotiff} file, a container format which specifies necessary metadata and the underlying image data in a lossless format such as \texttt{PNG}.
This poses the problem of having to look up which files that cover a given region of interest and merging these files together before processing them.

A simpler approach is to create a \textit{GDAL Virtual Format} file (\texttt{VRT}) which is a virtual dataset file referencing all the respective tiles and bands (GIS uses the term bands for what we would otherwise refer to as image channels).
In simple cases, a \texttt{VRT} file can be autogenerated with the \texttt{gdalbuildvrt} GDAL utility.

\begin{shellcode}
$ gdalbuildvrt raster.vrt ${raster_directory}/*.tif
\end{shellcode}
%
The resulting \texttt{vrt} file behaves like single merged file, and can be read and processed by most GIS tools, but in practice it is just a small XML file and does not require you to load the entire raster dataset into memory.

Using the same file format, we can also combine overlapping raster datasets by expanding the number of channels in the resulting raster.

\begin{shellcode}
$ gdalbuildvrt -resolution ${resolution} combined.vrt -separate ${vrt1} ${vrt2} 
\end{shellcode}
%
Where \texttt{-resolution} can be set to either \texttt{highest}, \texttt{lowest}, or \texttt{average}, depending on how you want to handle datasets with different resolutions.
This is how we will merge the aerial photography (RGB) data with the DSM data (Z), resulting in a single consistent ZRGB dataset.
The resulting \texttt{VRT} file will only contain the first band from each source file, and needs to be manually edited according to the VRT schema~\cite{vrt_schema_2015} in order to include the green and blue bands of the original RGB dataset.
