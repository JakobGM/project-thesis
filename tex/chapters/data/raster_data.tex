Raster data is simply scalar measurements imposed onto a grid with rectangular grid cells.

A color image, $I$, of width $\mathrm{W}$ and height $\mathrm{H}$, for instance, will contain three color channels, \textcolor{red}{red}, \textcolor{green}{green}, and \textcolor{blue}{blue}, (RGB) and can be represented by a three-dimensional array of size $\mathrm{W} \times \mathrm{H} \times \mathrm{3}$.
Each color channel in a given pixel is represented by an unsigned 8-bit integer, i.e. \\
\begin{equation*}
  I_{i, j, c} \in \{0, 1, ..., 255\},
  \hspace{2.5em}
  i = 0, ..., \mathrm{H} - 1,
  ~~
  j = 0, ..., \mathrm{W} - 1,
  ~~
  c = \mathrm{\textcolor{red}{R}, \textcolor{green}{G},\textcolor{blue}{B}}.
\end{equation*}

A lidar height map, $Z$, is likewise encoded as a single-channel grayscale image of size $\mathrm{W} \times \mathrm{H}$.
Each pixel is represented by a signed 32-bit floating point value, i.e.
\begin{equation*}
  Z_{i, j} \in \mathbb{R},
  \hspace{2.5em}
  i = 0, ..., \mathrm{H} - 1,
  ~~
  j = 0, ..., \mathrm{W} - 1.
\end{equation*}

These two raster types must be handled differently during data normalization due to their different value domains, which we will come back to in section \ref{sec:raster-normalization}.

% Keep the following section on the same page
\begin{minipage}{\textwidth}
  For GIS rasters we must define the spatial extent of the given image as well, specified by:
  \begin{itemize}[noitemsep]
    \item A coordinate system, for example UTM \texttt{32V}.
    \item The coordinate of the center of the upper left pixel, $I_{1, 1}$; the \textit{origin} $\vec{r}_0 = [x_0, y_0]^T$.
    \item The pixel step size, $\vec{\Delta} = [\Delta_x, \Delta_y]^T$, for example $[\SI{0.25}{\meter}, \SI{-0.25}{\meter}]^T$.
  \end{itemize}
\end{minipage}
\vspace{0.5em}

The pixel $I_{i, j, c}$ therefore represents a rectangle of width $\Delta_x$ and height $\Delta_y$ centered at the spatial coordinate $\vec{r}_0 + \vec{\Delta} \cdot [i, j]$, everything being interpreted in the given coordinate system.

Missing data in remote sensing rasters are specified by filling in a pre-defined \texttt{nodata} placeholder value.
For RGB data this is often set to $0$, resulting in a black pixel.
LiDAR rasters often use $\texttt{nodata} = -2^{127} \times (2 - 2^{-23}) \approx 3.4028234664 \times 10^{38}$, the smallest normal number representable by a single-precision floating point number.
Such \texttt{nodata} values may arise from measurement errors and regions which are outside the given covarage area of the dataset, and must be special-cased during data normalization, which we will come back to in section \ref{sec:raster-normalization}.
