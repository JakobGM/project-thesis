\section*{Introduction}
\addcontentsline{toc}{section}{Introduction}\markboth{Introduction}{Introduction}
\pagenumbering{arabic}  % Start using Arabic numbers again
\setcounter{page}{1}  % Reset page number
\fancyhead[R]{\footnotesize\thepage\ of\ \pageref{LastPage}}  % Start writing page x of y

\textit{Remote sensing} is the process of gathering information about an object without making physical contact, one such technology being \textit{aerial photography}.
Although aerial RGB photography is intuitively interpretable for humans, it is fundamentally two-dimensional.
\textit{LiDAR}, another remote sensing technology, is able to measure distances to object surfaces by directing a beam of light and measuring the time of arrival and wavelength of the ensuing reflection.
The resulting data can therefore be used to construct a three-dimensional spatial representation of the object of interest.
LiDAR has been applied in a wide array of fields such as meteorology~\cite{lidar_meteorology_1966}, forestry analysis~\cite{lidar_forestry_2000}, urban flood modelling~\cite{lidar_flood_2013}, and autonomous driving systems~\cite{lidar_self_driving_2018}.

One of the applications of LiDAR technology is the construction of \textit{digital surface models} (DSMs).
DSMs are grayscale images representing the earth's surface including all above-surface objects such as natural canopy and human-made objects.
In contrast, \textit{digital terrain models} (DTMs) represent the elevation of the \textit{bare} ground where all above-surface objects have been artificially removed.
While DTMs are often used in geographic and cartographic applications, DSMs can be used for localization and classification of objects above ground.

LiDAR data and aerial photography is usually provided by the respective cadastral authority in a given country.
Cadastral authorities are also responsible for keeping records of cadastral data such as cadastral plots, roads, and buildings.
The exact type and quality of this data varies substantially between countries and sometimes even between administrative regions in the same country.
This raises the question: \enquote{Can high-fidelity insights be inferred from otherwise low-fidelity geographic data?}.
\figref{fig:data-enchancement} shows an outline of the possible \enquote{data enhancements} which are of interest within this domain. %chktex 2

\begin{figure}[htb]
  \includegraphics[width=\linewidth]{data-enchancement}
  \caption{
    Classification of geographic data quality.
    The classifications reflect a general observed trend in data sets, and a given region may therefore not fit into exactly one of these categories.
    Some of the data types mentioned here will be described in~\secref{sec:data}.
  }%
  \label{fig:data-enchancement}
\end{figure}

The \textit{Norwegian Mapping and Cadastre Authority} (\textit{Statens Kartverk}) provides geographic data of uniquely high quality for the entirety of Norway.
This offers an opportunity to train supervised machine learning models on lower fidelity data in order to infer higher fidelity features.
Such models can then be applied in other regions where only low-fidelity data is available as a method of data enhancement.

\subsubsection*{Research questions}

A \textit{building outline} is a two-dimensional representation of building \enquote{footprint}.
Such data can be used for map annotations, flood risk analysis, and population density estimates, amongst other applications.
The identification of building outlines from remote sensing data can be formulated as a \textit{semantic segmentation task}, a heavily researched topic which has had many advances in the last decade.
Geographic data, such as building outlines, are formatted in an unsuitable way for direct machine learning, and must therefore be purposefully transformed and pre-processed.
The development of a data pipeline for geographic data is the first topic of research in this thesis.

\begin{description}
  \item[RQ1] How can geographic data representation be transformed into a format suitable for machine learning?
\end{description}

After having developed such a pipeline, the focus will be to develop a segmentation model for identifying building footprints with data from this pipeline.
The use of aerial photography and LiDAR data from the Norwegian municipality of Trondheim will be investigated, as well as the combination of these two data sources.

\begin{description}
  \item[RQ2] How can aerial photography and/or LiDAR data be used to predict accurate building outlines?
\end{description}

These research questions will lay down the ground work for my upcoming master's thesis where I will investigate the possibility of inferring roof surfaces represented as three-dimensional polygons from remote sensing data.
Three-dimensional representations of buildings can for example be used for urban planning purposes.
Another application, which incidentally prompted this research question in the first place, is the use of roof surface geometries to estimate the potential energy production of roof-mounted solar panel installations.

\subsubsection*{Thesis disposition}

We will start by providing an overview of the problem domain of image segmentation and the methods currently being applied in the field in~\secref{sec:segmentation}.
An introduction to the world of \textit{Geographic Information Systems} (GIS); the field which concerns itself with representing geographic data, will follow in~\secref{sec:data}.
\secref{sec:pre-processing} will describe how geographic data can be pre-processed in order to train accurate machine learning models.
The chosen model architecture, training procedure, and the final experimental results will be presented in~\secref{sec:experiments}.
