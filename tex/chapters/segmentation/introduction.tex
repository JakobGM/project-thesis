The field of \textit{computer vision} got started in the early 1970s \cite[p.~10]{computer_vision_history}.
Computer vision differs from the classical discipline of \textit{digital image processing} by concerning itself with the three-dimensional reconstruction of a scene from two-dimensional data \cite[p.~10]{computer_vision_history}.
Most of the early research in the field revolved around manually designed feature extraction and explicit processing techniques, but statistical techniques started to appear in the 1990s \cite[p.~15]{computer_vision_history}.
The statistical approach to the problem domain eventually morphed into the field of \textit{machine learning}, where most of the advances are made today \cite[p.~17]{computer_vision_history}.

In this section, we will start by describing the particular image recognition problem of interest for the our research question, namely \textit{semantic segmentation}.
Section \ref{sec:segmentation-metrics} will summarize the metrics used for evaluating the quality of a given image segmentation.
\textit{Convolution neural networks} (CNNs) are one of the most successful techniques applied to image segmentation problems \cite[p.~1]{image_recognition}, and will be described in section \ref{sec:cnn}.
State-of-the-art CNN architectures for image segmentation will be listed in section \ref{sec:cnn-architectures}, and existing work within the field of segmentation using remote sensing data is described in section \ref{sec:remote-sensing-research}.
