Image recognition seeks to answer three questions for any given image \cite[p.~1]{image_recognition}:

\begin{enumerate}[nosep]
  \item \textbf{Identification:} Does the image contain any object of interest?
  \item \textbf{Localization:} Where in the image are the objects situated?
  \item \textbf{Classification:} To which categories do the objects belong to?
\end{enumerate}

We will concern ourselves with only one object category (class) at any time, that class being buildings, and will simplify the following theory accordingly with this simplification in mind.
The localization and classification of objects in a given image can be performed at different granularity levels, as shown in \figref{fig:segmentation-types}.

\begin{figure}[htb]
  \includegraphics[width=\linewidth,trim={0 4cm 0 3.4cm},clip]{segmentation-types}
  \caption{
    Different granularities for single-class building localization, using the Trondheim 2017 data set.
    Bounding box regression is shown on the left, semantic segmentation in the middle, and instance segmentation on the right.
  }
  \label{fig:segmentation-types}
\end{figure}
\newpage

\textit{Bounding box regression} concerns itself with finding the smallest possible rectangle which envelopes the object of interest.
The rectangle sides may either by oriented parallel to the axis directions, or rotated in order to attain the smallest possible envelope.
For object shapes which are not perfectly rectangular, the bounding box will therefore necessarily contain pixels that are \textit{not} part of the object itself.

\textit{Segmentation} rectifies this issue by classifying each pixel in the image independently, i.e. \textit{pixel-wise} classification, referred to as a classification \textit{mask}.
While \textit{instance} segmentation distinguishes between pixels of the same class but part of different objects, \textit{segmentation} does not make this distinction.
Since a bounding box can be directly derived from a semantic segmentation mask, and a semantic segmentation mask can be directly derived from instance segmentation mask; the problem complexity of these tasks are as follows:
%
\begin{equation*}
  \text{Bounding box regression}
  <
  \text{Semantic segmentation}
  <
  \text{Instance segmentation}
\end{equation*}
%
An image with $C$ color channels, and of width $W$ and height $H$, is represented by a tensor $X \in \mathbb{R}^{W \times H \times C}$.
This is somewhat simplified, but we will give a more nuanced description in section \ref{sec:raster-data}.
Single-class semantic segmentation can therefore be formalized as constructing a binary predictor $\hat{f}$ of the form:
%
\begin{equation*}
  \hat{f}: \mathbb{R}^{W \times H \times C} \rightarrow \mathbb{B}^{W \times H}, \hspace{2em} \text{where } \mathbb{B} \defeq \{0, 1\}.
\end{equation*}
%
Where $\mathbb{B}^{W \times H}$ denotes a boolean matrix, $1$ indicating that the pixel is part of the object class of interest, and $0$ indicates the opposite.
In practice, however, statistical models will often predict a pixel-wise class \textit{confidence} in the continuous domain $[0, 1]$,
%
\begin{equation*}
  \tilde{f}: \mathbb{R}^{W \times H \times C} \rightarrow [0, 1]^{W \times H},
\end{equation*}
%
but a binary predictor can be easily constructed by choosing a suitable threshold, $T$, for which to distinguish positive predictions from negative ones
%
\begin{equation*}
  \hat{f}(X) = \tilde{f}(X) > T, \hspace{2em} X \in \mathbb{R}^{W \times H \times C}.
\end{equation*}
%
The choice of the exact threshold value, $T$, will influence the resulting \textit{sensitivity} and \textit{specificity} metrics of the model, terms which will be explained in the upcoming section \ref{sec:segmentation-metrics}.
