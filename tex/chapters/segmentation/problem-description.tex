Image recognition seeks to answer three questions for any given image~\cite{image_recognition}:

\begin{enumerate}[nosep]
  \item \textbf{Identification:} Does the image contain any object of interest?
  \item \textbf{Localization:} Where in the image are the objects situated?
  \item \textbf{Classification:} To which categories do the objects belong to?
\end{enumerate}

We will concern ourselves with only one object category (class) at any time, that class being building footprints, and will simplify the following theory accordingly with this simplification in mind.
The localization and classification of objects in a given image can be performed at different granularity levels, as shown in~\figref{fig:segmentation-types}.

\begin{figure}[htb]
  \includegraphics[width=\linewidth,trim={0 4cm 0 3.4cm},clip]{segmentation-types}
  \caption{
    Different granularities for single-class building localization, using the Trondheim 2017 data set.
    Bounding box regression is shown on the left, semantic segmentation in the middle, and instance segmentation on the right.
  }%
  \label{fig:segmentation-types}
\end{figure}

\textit{Bounding box regression} concerns itself with finding the smallest possible rectangles which envelopes the objects of interest.
The sides of the rectangles may either by oriented parallel to the axis directions, or rotated in order to attain the smallest possible envelope.
The bounding box will therefore necessarily contain pixels that are not part of the object itself whenever the object shape is not perfectly rectangular.

\textit{Semantic segmentation} rectifies this issue by classifying each pixel in the image independently, i.e. \textit{pixel-wise} classification, producing a so-called classification \textit{mask}.
\textit{Instance segmentation} distinguishes between pixels belonging to different objects of the same class, while \textit{semantic segmentation} does not make this distinction.
Since a bounding box can be directly derived from a semantic segmentation mask, and a semantic segmentation mask can be directly derived from instance segmentation mask; the problem complexity of these tasks are as follows:
%
\begin{equation*}
  \text{Bounding box regression}
  <
  \text{Semantic segmentation}
  <
  \text{Instance segmentation}
\end{equation*}
%
An image of width $W$ and height $H$ consisting of $C$ channels is represented by a $W \times H \times C$ tensor, $X \in \mathbb{R}^{W \times H \times C}$.
This is somewhat simplified, but we will give a more nuanced description in~\secref{sec:raster-data}.
Single-class semantic segmentation can therefore be formalized as constructing a binary predictor $\tilde{f}$ of the form:
%
\begin{equation*}
  \tilde{f}: \mathbb{R}^{W \times H \times C} \rightarrow \mathbb{B}^{W \times H}, \hspace{2em} \mathbb{B} \defeq \{0, 1\}.
\end{equation*}
%
Where $\mathbb{B}^{W \times H}$ denotes a boolean matrix, $1$ indicating that the pixel is part of the object class of interest, and $0$ indicates the opposite.
In practice, however, statistical models will often predict a pixel-wise class \textit{confidence} in the continuous domain $[0, 1]$,
%
\begin{equation*}
  \hat{f}: \mathbb{R}^{W \times H \times C} \rightarrow {[0, 1]}^{W \times H},
\end{equation*}
%
but a binary predictor can be easily constructed by choosing a suitable threshold, $T$, for which to distinguish positive predictions from negative ones
%
\begin{equation*}
  \tilde{f}(X) = \hat{f}(X) > T, \hspace{2em} X \in \mathbb{R}^{W \times H \times C}.
\end{equation*}
%
The choice of the threshold value $T$ will affect the resulting \textit{sensitivity} and \textit{specificity} metrics of the model predictions, metrics which will be explained in the upcoming~\secref{sec:segmentation-metrics}.
