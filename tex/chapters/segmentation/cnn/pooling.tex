The conventional last step in a given CNN layer is the application of a \textit{downsampling} operation, most often referred to as \textit{pooling}.
As with convolution this operation has biological inspirations as it is inspired by a model of the mammalian visual cortex~\cite[p.~966]{visint-cnn}.
The reduction in spatial resolution is considered to be one of the main reasons for why CNNs portray a high degree of translational invariance~\cite{cnn-translational-invariance}.
As with moving convolution, pooling is implemented by moving a receptive field of size $>1$ (typically $2 \times 2$) over the activations and mapping these values to a lower dimensional space.
There are several different ways to define such a mapping, the two most common being \textit{max pooling} and \textit{average pooling}, which respectively retrieve the maximum value and average value from the receptive field.
The former is exemplified in~\figref{fig:max-pooling}.

\begin{figure}[htb]
  The last layer in a given CNN block is conventionally a \textit{downsampling} operation, most often referred to as a \textit{pooling} layer.
As with convolution, this operation has biological influences as it is inspired by a model of the mammalian visual cortex~\cite[p.~966]{visint-cnn}.
The reduction in spatial resolution is considered to be one of the main reasons for why CNNs portray a high degree of translational and rotational invariance~\cite{cnn-translational-invariance}.
As with moving convolution, pooling is implemented by moving a receptive field of size greater than $1$, typically $2 \times 2$, over the activations and mapping these values into a lower dimensional space.
There are several different ways to define such a mapping, the two most common being \textit{max pooling} and \textit{average pooling}, which respectively retrieve the maximum value and average value from the receptive field.
The former is exemplified in~\figref{fig:max-pooling}.

\begin{figure}[htb]
  The last layer in a given CNN block is conventionally a \textit{downsampling} operation, most often referred to as a \textit{pooling} layer.
As with convolution, this operation has biological influences as it is inspired by a model of the mammalian visual cortex~\cite[p.~966]{visint-cnn}.
The reduction in spatial resolution is considered to be one of the main reasons for why CNNs portray a high degree of translational and rotational invariance~\cite{cnn-translational-invariance}.
As with moving convolution, pooling is implemented by moving a receptive field of size greater than $1$, typically $2 \times 2$, over the activations and mapping these values into a lower dimensional space.
There are several different ways to define such a mapping, the two most common being \textit{max pooling} and \textit{average pooling}, which respectively retrieve the maximum value and average value from the receptive field.
The former is exemplified in~\figref{fig:max-pooling}.

\begin{figure}[htb]
  The last layer in a given CNN block is conventionally a \textit{downsampling} operation, most often referred to as a \textit{pooling} layer.
As with convolution, this operation has biological influences as it is inspired by a model of the mammalian visual cortex~\cite[p.~966]{visint-cnn}.
The reduction in spatial resolution is considered to be one of the main reasons for why CNNs portray a high degree of translational and rotational invariance~\cite{cnn-translational-invariance}.
As with moving convolution, pooling is implemented by moving a receptive field of size greater than $1$, typically $2 \times 2$, over the activations and mapping these values into a lower dimensional space.
There are several different ways to define such a mapping, the two most common being \textit{max pooling} and \textit{average pooling}, which respectively retrieve the maximum value and average value from the receptive field.
The former is exemplified in~\figref{fig:max-pooling}.

\begin{figure}[htb]
  \input{tikz/pooling.tex}
  \caption{%
    Example of a \textit{max-pooling} operation with a receptive field of size $2 \times 2$ and an identical stride size.
    The receptive field is shown in \textcolor{orange}{orange} and the respective pooled output is shown in \textcolor{green}{green}.
  }%
  \label{fig:max-pooling}
\end{figure}

As can be seen in~\figref{fig:max-pooling}, using a receptive field and stride of size $2 \times 2$ will yield a downsampled image with one quarter as many pixels as the original input.

  \caption{%
    Example of a \textit{max-pooling} operation with a receptive field of size $2 \times 2$ and an identical stride size.
    The receptive field is shown in \textcolor{orange}{orange} and the respective pooled output is shown in \textcolor{green}{green}.
  }%
  \label{fig:max-pooling}
\end{figure}

As can be seen in~\figref{fig:max-pooling}, using a receptive field and stride of size $2 \times 2$ will yield a downsampled image with one quarter as many pixels as the original input.

  \caption{%
    Example of a \textit{max-pooling} operation with a receptive field of size $2 \times 2$ and an identical stride size.
    The receptive field is shown in \textcolor{orange}{orange} and the respective pooled output is shown in \textcolor{green}{green}.
  }%
  \label{fig:max-pooling}
\end{figure}

As can be seen in~\figref{fig:max-pooling}, using a receptive field and stride of size $2 \times 2$ will yield a downsampled image with one quarter as many pixels as the original input.

  \caption{%
    Example of a \textit{max-pooling} operation with a receptive field of size $2 \times 2$ and an identical stride size.
    The receptive field is shown in \textcolor{orange}{orange} and the respective pooled output is shown in \textcolor{green}{green}.
  }%
  \label{fig:max-pooling}
\end{figure}

As can be seen in~\figref{fig:max-pooling}, using a receptive field and stride of size $2 \times 2$ will yield a downsampled image with one quarter as many pixels as the original input.

\todo{Anything more to be said here?}
