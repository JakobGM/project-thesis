The conventional last step in a given CNN layer is the application of a downsampling operation, referred to as \textit{pooling}.
Yet again it has a biological motivation as it is inspired by the model of the visual cortex of the mammalian brain.
There are several different methods of downsampling the resolution.
As with moving convolution, pooling is implemented by moving a receptive field of size $>1$ (typically $2 \times 2$) over the activations and mapping these values to a lower dimensional space.
There are several different ways to define such a mapping, the two most common being \textit{max pooling} and \textit{average pooling}, which respectively retrieve the maximum value and average value in the receptive field.
Using a receptive field of size $2 \times 2$ and an identical stride size will therefore yield a downsampled image with one quarter as many pixels as the original input.
