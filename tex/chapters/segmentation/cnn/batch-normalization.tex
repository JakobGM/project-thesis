Reparametrization of earlier layers during the training of a deep neural network results in a distributional change in the feature layer passed onto the subsequent layers.
This forces subsequent layers to adapt to the new \enquote{distributional circumstances}, slowing training convergence in the process.
This phenomenon, referred to as \textit{internal covariate shift}, was identified in a paper~\cite{batch-normalization} by \citeauthor{batch-normalization}~(\citeyear{batch-normalization}) where they propose a method called \textit{batch normalization} in order to remedy this issue.
Propose we have a layer activation $\vec{x}$ consisting of $d$ dimensions, i.e. $\vec{x} = (x^{(1)}, \ldots, x^{(d)})$.
At first we normalize each feature dimension, $k$, independently

\begin{equation*}
  \widehat{x}^{(k)}
  =
  \frac{
    x^{(k)} - \E{x^{(k)}}
  }{
    \sqrt{\Var{x^{(k)}} + \epsilon}
  },
  \tag{Statistical Normalization}
\end{equation*}

where $\E{\cdot}$ and $\Var{\cdot}$ are respectively sample means and sample variances over the current mini-batch, and $\epsilon$ is added for numerical stability.
The result of this statistical normalization is a feature layer where every dimension has a mean of $0$ and a variance of $1$.
The internal covariate shift has been practically eliminated as a result.

This type of normalization alone may not be optimal in all cases, though, and is best explained with an example.
Assume a set of layer activations $\vec{x}$ to be distributed symmetrically across the mean, and set the activation function of the subsequent layer to the $\relu(x) = \max(x,0)$ activation function.
After normalization, 50\% of the values are expected to be negative, and are therefore truncated to $0$.
This informational loss may be suboptimal for the given network layer and must be accounted for.
For this reason, we introduce two additional trainable parameters for each feature dimension, $\gamma^{(k)}$ and $\beta^{(k)}$, and apply a second normalization step

\begin{equation*}
  y^{(k)} = \gamma^{(k)} \widehat{x}^{(k)} + \beta^{(k)}.
  \tag{Trainable Normalization}
\end{equation*}

The intent is to learn the values for the shift, $\beta^{(k)}$, and scaler, $\gamma^{(k)}$, which restores the representative power of the given layer \textit{after} the statistical normalization.
