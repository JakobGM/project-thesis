We now construct a model which uses both LiDAR and RGB data in combination in order to produce predictions.
The training procedure of the combined data model is shown in~\figref{fig:rgb-lidar-training}, and the training procedure of the LiDAR model has been included for comparison.

\begin{figure}[H]
  \centering
  \includegraphics[width=0.495\linewidth]{metrics/with_rgb+without_rgb-train+validation-iou}
  \includegraphics[width=0.495\textwidth]{iou_distribution/with_rgb}
  \caption{%
    \textbf{Left --} IoU evaluation during training of U-Net models for 89 epochs.
    \textcolor{blue}{Blue} indicates the model using just LiDAR data, while \textcolor{orange}{orange} is used to indicate the combined data model (RGB + LiDAR).
    % Solid lines indicate validation split metrics, while dashed lines indicate training split metrics.
    % The epochs yielding the best validation IoU for each model are annotated as solid circles with respective colors.
    \textbf{Right --} Test IoU distribution of combined data model, left \SI{5}{\percent} of data cropped.
    See caption of~\figref{fig:iou-distribution-explanation} for detailed description.
  }%
  \label{fig:rgb-lidar-training}
\end{figure}
\vspace{-\baselineskip}
The median performing test prediction is shown in~\figref{fig:rgb-lidar-median}.
%
\begin{figure}[H]
  \centering
  \includegraphics[width=0.6\linewidth]{predictions/with_rgb-24017-0}  % chktex 8
  \caption{%
    Median IoU prediction from the test set using both remote sensing data types, RGB and LiDAR.
  }%
  \label{fig:rgb-lidar-median}
\end{figure}

Most predictions produced by the combined model show the same behaviour and general weaknesses as the model based solely on LiDAR data.
Although the LiDAR model outperformed the RGB model in every aggregate metric, indicating that RGB is a worse predictor than LiDAR, using RGB \emph{in addition} to LiDAR as input data still increases the test performance of the resulting model.
When adding RGB to to LiDAR model, the mean IoU on the test set improves from \num{0.934} to \num{0.938}, not an insignificant improvement.
A tile-by-tile comparison of the LiDAR model and the combined data model, similar to the comparison presented in~\figref{fig:correlation-explanation}, is shown in~\figref{fig:lidar-combined-correlation}.
The combined data model outperforms the LiDAR model in \SI{61.2}{\percent} of all test cases.

\begin{figure}[H]
  \centering
  \includegraphics[width=0.9\linewidth]{metric_correlation/without_rgb+with_rgb+iou}
  \caption{%
    Scatter plot showing the correlation between the evaluation metric performance of two models, LiDAR vs.\ RGB + LiDAR\@.
    The combined data model is shown along the vertical axis, while the model using just LiDAR data is shown along the horizontal axis.
    See caption of~\figref{fig:correlation-explanation} for detailed figure explanation.
  }%
  \label{fig:lidar-combined-correlation}
\end{figure}

In what way does the combined data model outperform the LiDAR model?
The combined model usually produces predictions almost identical to the LiDAR model, but in the minority of the cases where the RGB model performs \emph{better} than the LiDAR model, the combined model seems to mimic the RGB model rather than the LiDAR model.
Previously, in~\figref{fig:rgb-better-than-lidar}, we presented two test tiles where the RGB model evaluated \emph{better} than the LiDAR model.
The combined data model predictions on the same test tiles are presented in~\figref{fig:rgb-selection}.

\begin{figure}[H]
  \centering
  \includegraphics[width=0.49\textwidth]{predictions/with_rgb-26819-1}  % chktex 8
  \textcolor{gray}{\vrule}
  \includegraphics[width=0.49\textwidth]{predictions/with_rgb-9702-0}  % chktex 8
  \caption{%
    Predictions on test tiles using the model trained on both LiDAR data and RGB data in combination.
    The individual data model predictions using the same tiles are given in~\figref{fig:rgb-better-than-lidar}.
  }%
  \label{fig:rgb-selection}
\end{figure}

The observed pattern is that the combined data model usually imitates the LiDAR-only model, only deviating whenever the RGB data is more descriptive than the LiDAR data.
\figref{fig:rgb-help} substantiates this interpretation of how the combined data model improves upon the LiDAR model.

\begin{figure}[H]
  \begin{tikzpicture}
    \node[anchor=south west,inner sep=0] (image) at (0,0) {\hspace{2em}\includegraphics[width=0.66\linewidth]{rgb-helps}};
    \begin{scope}[x={(image.south east)},y={(image.north west)},overlay,text width=3cm,font=\footnotesize]
      \node (first) at (1, 0.775) {};
      \node (second) at (0, 0.775) {};
      \node (third) at (0, 0.32) {};
      \node (fourth) at (1, 0.32) {};
      \node[draw] at (first) {\textbf{1\textsuperscript{st} quadrant}\\Both combined data model \emph{and} RGB model perform better than the LiDAR model.};
      \node[draw] at (second) {\textbf{2\textsuperscript{nd} quadrant}\\Combined data model performs better than the LiDAR model, but RGB model performs worse.};
      \node[draw] at (third) {\textbf{3\textsuperscript{rd} quadrant}\\Both combined data model \emph{and} RGB model perform worse than the LiDAR model.};
      \node[draw] at (fourth) {\textbf{4\textsuperscript{th} quadrant}\\Combined data model performs worse than the LiDAR model, but the RGB model performs better.};
    \end{scope}
  \end{tikzpicture}
  \caption{%
    Scatter plot of points $(x_i, y_i)$, one point for each test tile, $i$.
    The $x$-coordinate is the difference between the RGB model's IoU evaluation and the LiDAR model's IoU evaluation, i.e.\ how much better RGB performs than LiDAR.
    The $y$-coordinate is the difference between the combined data model's IoU evaluation and the LiDAR model's evaluation, i.e.\ how much better the combined data performs compared to using LiDAR in isolation.
    The median $y$-coordinate has been calculated for non-overlapping bins of width $\Delta x = \num{0.1}$ and have been annotated in \textcolor{orange}{orange}.
  }%
  \label{fig:rgb-help}
\end{figure}

Each scatter point in~\figref{fig:rgb-help} represents a geographic tile in the test split, and the coordinates $(x_i, y_i)$ are derived from linear combinations of the IoU metrics of the LiDAR model, RGB model, and the combined data model.
The $x$-coordinate is the difference between the IoU of the RGB model prediction and the IoU of the LiDAR model prediction.
In other words, whenever $x > 0$, the RGB model performs better than the LiDAR model.
The $y$-coordinate is the difference between the combined data model's IoU metric and the LiDAR model's IoU metric.
Whenever $y > 0$, the combined data model performs better than the model based on just LiDAR data.
Consider a hypothetical model which is able to omnisciently select between the RGB and LiDAR model based on which maximizes the resulting evaluation performance; such a model would produce scatter points according to the relation $y = \max(0, x)$ in~\figref{fig:rgb-help}.
The combined data model does in fact produce predictions somewhat in accordance to this relation as illustrated by the \enquote{moving median} in~\figref{fig:rgb-help}.
