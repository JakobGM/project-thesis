We will now investigate how the normalization of LiDAR raster elevation values influences the predictive performance of the resulting model.
The \enquote{\texttt{nodata}-aware local min-max normalization} method described in~\algref{alg:local-min-max-scaling} on page~\pageref{alg:local-min-max-scaling} will be simply referred to as \enquote{dynamic scaling} as it scales each elevation tile individually to the $[0, 1]$ domain.
Likewise, the \enquote{\texttt{nodata}-aware metric normalization} method described in~\algref{alg:metric-normalization} on page~\pageref{alg:metric-normalization} will be simply referred to as \enquote{constant scaling} as it always scales by a constant factor $\gamma^{-1}$.
The training procedures of two models employing these two different LiDAR normalization methods are shown in~\figref{fig:normalization-training}.
The global scaler $\gamma$, as specified in~\algref{alg:metric-normalization}, is chosen to be 30.
Small values for this global scaler, $\gamma < 10$, results in no training convergence whatsoever, while large enough values have been shown to not differ significantly in performance.

\begin{figure}[H]
  \centering
  \includegraphics[width=0.495\linewidth]{metrics/lidar_metric_normalization+without_rgb-validation-iou}
  \includegraphics[width=0.495\textwidth]{iou_distribution/lidar_metric_normalization}
  \caption{%
    \textbf{Left --} Training procedure of U-Net LiDAR model using two different normalization methods.
    The model using \enquote{dynamic scaling} as specified in~\algref{alg:local-min-max-scaling} is shown in \textcolor{blue}{blue}, while the model using \enquote{constant scaling} with $\gamma = 30$ as specified in~\algref{alg:metric-normalization} is shown in \textcolor{orange}{orange}.
    \textbf{Right~--}~Test IoU distribution \enquote{dynamic scaling} model, left \SI{4}{\percent} of data cropped.
    See caption of~\figref{fig:iou-distribution-explanation} for detailed description.
  }%
  \label{fig:normalization-training}
\end{figure}

\figref{fig:normalization-training} shows a small improvement of using the constant scaling over the dynamic method, an improvement of validation IoU from \num{0.9366} to \num{0.9393}, although not much can be concluded from this figure alone.
A comparison over the training and test splits of the two models is presented in~\figref{fig:normalization-correlation}.

\begin{figure}[H]
  \centering
  \includegraphics[width=0.9\linewidth]{metric_correlation/without_rgb+lidar_metric_normalization+iou}
  \caption{%
    Scatter plot showing the correlation between models using different LiDAR normalization methods, IoU of the \enquote{constant scaling} model annotated along the vertical axis while the IoU of the \enquote{dynamic scaling} model is annotated along the horizontal axis.
    See caption of~\figref{fig:correlation-explanation} for detailed figure explanation.
  }%
  \label{fig:normalization-correlation}
\end{figure}

The \enquote{constant scaling} normalization method does in fact perform better than the \enquote{dynamic scaling} normalization method over all three sample splits: training, validation, and test.
When the normalization method is changed to the constant scaling method, the mean test IoU metric improves from \num{0.922} to \num{0.929}, and 63\% of the test cases perform better.
Some of the test cases where the constant scaling model performs better are tiles containing large, flat areas that are \emph{not} part of a roof, for example cadastral plots situated along the shoreline as presented in~\figref{fig:constant-better-than-dynamic}.
Although we have established that constant scaling is an overall improvement over dynamic scaling, we have not been able to identify any further problem characteristics where constant scaling generally performs better.
It was hypothesized that constant scaling would outperform dynamic scaling whenever the elevation range within a given tile became very small or very large, this being due to the lossy compression of dynamic scaling forcing all elevation values into the $[0, 1]$ value range.
Such an effect has not been observed.
The largest improvement of the constant scaling model over the dynamic scaling model has largely been due to an improvement in recall from \SI[round-mode=places,round-precision=2]{88.91553113410536}{\percent} to \SI[round-mode=places,round-precision=2]{89.32751737113643}{\percent}, while precision only improved from \SI[round-mode=places,round-precision=2]{89.20540297068338}{\percent} to \SI[round-mode=places,round-precision=2]{89.22041798211237}{\percent}.

\begin{figure}[H]
  \centering
  \marginlabel{\includegraphics[width=0.9\linewidth]{prediction_improvement/without_rgb+lidar_metric_normalization/21414+0/worst.pdf}}
  \marginlabel{\includegraphics[width=0.9\linewidth]{prediction_improvement/without_rgb+lidar_metric_normalization/21414+0/best.pdf}}
  \caption{%
    Geographic test tiles where the constant scaling model (top) performs much better than the dynamic scaling model (bottom).
  }%
  \label{fig:constant-better-than-dynamic}
\end{figure}
