\topic{Training procedure}

\num{58559} input tiles, including both aerial photography (RGB) and digital surface model (LiDAR elevation), have been preprocessed and persisted to disk.
This sample space is further split into \num{41018} training tiles, \num{8788} validation tiles, and \num{8753} test tiles, yielding a customary a 70\% / 15\% / 15\% training--validation--testing split when rounded to the nearest percent.
The training data is randomly shuffled and augmented at the beginning of each epoch in order to reduce overfitting.
The data augmentation consists of a random application of horizontal and/or vertical flipping in addition to a rotation by a random integer multiple of 90 degrees.
The training data is subsequently grouped into batches of size 16 before applying the Adam optimizer.
Training is continued until observed convergence by the use of the IoU evaluation of the validation split.
The weights corresponding to the epoch yielding the best validation metric is used as the final model parametrization.
\todo{Add figure for data augmentation?}

\topic{Software}

The source code used in order to produce and present the results in this paper is openly available at \url{https://github.com/JakobGM/project-thesis}.
Most of the source code is written in Python as it arguably has the best software ecosystem for both GIS workflows \emph{and} deep learning.
This project would not have been possible if not for the vast array of high quality open source software available, especially within the GIS ecosystem.
The Geospatial Data Abstraction Library (GDAL) \cite{dep:gdal} has been extensively used in order to process GIS data, and the python wrappers for GDAL, Rasterio \cite{dep:rasterio} for raster data and Fiona \cite{dep:fiona} for vector data, are central building blocks of the data processing pipeline.
Further, Numpy \cite{dep:numpy}, Shapely \cite{dep:shapely}, scikit-learn \cite{dep:sklearn} / scikit-image \cite{dep:sklearn}, and GeoPandas \cite{dep:geopandas} have been used in order to shape the data into a final format suitable for machine learning purposes.
The machine learning framework of choice has been the new 2.0 release of TensorFlow \cite{dep:tensorflow}, most of the modelling code having been written with the declarative Keras API.
This is not an exhaustive list of all dependencies, but a complete list of software dependencies and a reproducible Docker \cite{dep:docker} image is provided with the source code for this project.

\topic{Hardware}

All numerical experiments have been performed by a desktop class computer with the following relevant technical specifications:

\begin{itemize}[nosep]
  \item \textbf{Processor:} \textit{AMD Ryzen 9 3900X}. \\
    12 cores / 24 threads, \SI{3.8}{\giga\hertz} base clock / \SI{4.6}{\giga\hertz} boost clock.
  \item \textbf{Graphics card:} \textit{MSI GeForce 2070 Super}. \\
    \SI{8}{\giga\byte} GDDR6 VRAM, \SI{1605}{\mega\hertz} clock speed, \SI{9.062}{\tera\flops} 32-bit performance.
  \item \textbf{Memory:} \textit{Corsair Vengeance LPX DDR4 \SI{3200}{\mega\hertz} 32GB}.
  \item \textbf{Storage:} \textit{Intel 660p 1TB M.2 SSD}. \\
    Up to \SI{1800}{\mega\byte\per\second} read and write speed.
\end{itemize}

With a batch size of 16, each batched training step takes \SI{218}{\milli\second} of computation, resulting in approximately \SI{14}{\milli\second} per sample image tile.
With \num{2563} training batches, each training epoch requires approximately \SI{9.5}{\minute}, but with \num{549} additional validation batch evaluations each epoch uses approximately \num{12.25} minutes from end to end.
Most experiments have been trained for 90 epochs, hence requiring altogether approximately 18 hours and 20 minutes of training.
\todo{Clean up this section and add information about preprocessing.}
