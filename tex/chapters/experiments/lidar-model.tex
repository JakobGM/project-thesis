A model using only LiDAR data is trained, and the training procedure is summarized in~\figref{fig:lidar-training}.
\begin{figure}[H]
  \centering
  \includegraphics[width=0.9\linewidth]{metrics/without_rgb-train+validation-iou}
  \caption{%
    Training of U-Net model for 89 epochs, using LiDAR data.
    The training epochs are given along the horizontal axis, while the end-of-epoch IoU evaluations are given along the vertical axis.
    Validation split IoU is shown as a \textcolor{blue}{blue} solid line, while the training split IoU is shown as a \textcolor{blue}{blue} dashed line.
    The epoch yielding the best validation IoU is annotated as a solid \textcolor{blue}{blue} circle, in this case the 87th epoch with a validation IoU of \num{0.9366}.
  }%
  \label{fig:lidar-training}
\end{figure}

\todo{quantify exactly how much better LiDAR is compared to RGB}.
As before we plot the median test split prediction in~\figref{fig:lidar-median-prediction}.

\begin{figure}[H]
  \centering
  \includegraphics[width=\linewidth]{predictions/without_rgb-19872-0}  % chktex 8
  \caption{%
    LiDAR model median IoU prediction from the test set.
    The left panel shows the single-channel LiDAR input provided to the model.
    The remaining two tiles are identical to previous prediction plots; see caption of~\figref{fig:rgb-explanation} for detailed explanation.
  }%
  \label{fig:lidar-median-prediction}
\end{figure}

The LiDAR model has a great comparative advantage over the RGB model in that it manages to predict straight edges with a great degree of confidence and accuracy.
Whenever the LiDAR models fails it often includes and/or excludes a \enquote{well-defined} region and as a result still produces properly shaped building outlines.
The erroneous inclusion of an extended roof overhang over a front door or the exclusion of a small and low annex are two examples of commonly observed well-behaved failures.

While it has been established that the LiDAR model outperforms the RGB model \emph{in aggregate}, it is still of interest to compare these two models on a more case-by-case basis.
The two models are compared tile-by-tile in the IoU scatter plot presented in~\figref{fig:rgb-lidar-correlation}.

\begin{figure}[H]
  \centering
  \includegraphics[width=0.9\linewidth]{metric_correlation/without_rgb+only_rgb+iou}
  \caption{%
    Scatter plot showing the correlation between the evaluation metric performance of two models, LiDAR vs.\ RGB\@.
    Each \textcolor{blue}{blue} scatter point $(x_i, y_i)$ corresponds to a given tile, $i$, where the $x$-coordinate is the IoU metric of the LiDAR model prediction and the $y$-coordinate is the IoU metric of the RGB model prediction for that given tile.
    Tiles belonging to the train split are shown in the left half while the tiles belonging to the test split are shown in the right.
    The horizontal dashed lines in \textcolor{orange}{orange} indicate the \emph{mean} IoU metric of the RGB model for the respective splits, while the vertical dashed lines in \textcolor{green}{green} indicate the \emph{mean} IoU for the LiDAR model.
    Diagonal \textcolor{black}{black} lines indicates $x = y$, and the arrows with accompanying percentages indicate the fraction of points above and below this line.
  }%
  \label{fig:rgb-lidar-correlation}
\end{figure}

If the RGB and LiDAR models were indistinguishable in metric evaluation, the scatter points would be entirely situated along the diagonal black lines in~\figref{fig:rgb-lidar-correlation}, which is not the case here due to LiDAR outperforming the RGB model.
While LiDAR is on average better than RGB, RGB still beats LiDAR in about \SI{14}{\percent} of the test cases.
This may be partly caused by the randomness introduced into the training procedure, and thus the final model parametrization, but may be an early indication of a heterogeneous data model using both LiDAR \emph{and} RGB might outperform both these homogeneous data models.
Another thing to notice is how discrepancy between the train split and test split when it comes to the mean IoU metric is much larger for the RGB model (\num{0.04}) than for the LiDAR model (\num{0.026}).
Besides just outperforming the RGB model, this may also indicate that the LiDAR model generalizes better than the RGB model.

In~\figref{fig:lidar-fundamental-issues} we present the prediction of the LiDAR model on the same geographic tile as the RGB model prediction presented in~\figref{fig:rgb-fundamental-issues}.

\begin{figure}[H]
  \centering
  \marginlabel{\includegraphics[width=\linewidth]{predictions/without_rgb-2177-1}}  % chktex 8
  \marginlabel{\includegraphics[width=\linewidth]{predictions/without_rgb-2551-0}}  % chktex 8
  \caption{%
    LiDAR model prediction on same input as presented in~\figref{fig:rgb-fundamental-issues}.
  }%
  \label{fig:lidar-fundamental-issues}
\end{figure}

The LiDAR model predictions, \marginref{fig:lidar-fundamental-issues}{a} and \marginref{fig:lidar-fundamental-issues}{b}, demonstrates the same issues as seen with the RGB model, namely vanishingly small segmentation masks and erroneous ground truths.
This comes as no surprise, especially since these are considered independent of RGB data and its properties.
What about the RGB model issues considered intrinsic to the nature of RGB data, as exemplified in~\figref{fig:rgb-prediction-issues}; can these be remedied by the LiDAR model?
\figref{fig:lidar-corrected-rgb} presents the LiDAR model predictions on the same geographic tiles as used by the RGB model in~\figref{fig:rgb-prediction-issues}. % chktex 2

\begin{figure}[H]
  \centering
  \includegraphics[width=\linewidth]{predictions/without_rgb-29430-6}  % chktex 8
  \includegraphics[width=\linewidth]{predictions/without_rgb-31479-0}  % chktex 8
  \includegraphics[width=\linewidth]{predictions/without_rgb-45783-1}  % chktex 8
  \includegraphics[width=\linewidth]{predictions/without_rgb-8117-3}  % chktex 8
  \caption{%
    Fixes RGB issues.
  }%
  \label{fig:lidar-corrected-rgb}
\end{figure}

\begin{figure}[H]
  \centering
  \includegraphics[width=0.9\linewidth]{prediction_improvement/only_rgb+without_rgb/22880+1/worst.pdf}
  \includegraphics[width=0.9\linewidth]{prediction_improvement/only_rgb+without_rgb/22880+1/best.pdf}
  \rule[1ex]{\textwidth}{.5pt}
  \includegraphics[width=0.9\linewidth]{prediction_improvement/only_rgb+without_rgb/21925+3/worst.pdf}
  \includegraphics[width=0.9\linewidth]{prediction_improvement/only_rgb+without_rgb/21925+3/best.pdf}
  \caption{%
    Best improvement from RGB to LiDAR.
    (Top) Best improvement without area filter. Buildings along edges are usually more tricky for RGB model.
    \\
    (Bottom) Next best improvement from RGB to LiDAR without above average mask value.
    The best has already been shown in an earlier figure.
  }%
  \label{fig:lidar-better-than-rgb}
\end{figure}

\begin{figure}[H]
  \centering
  \includegraphics[width=0.9\linewidth]{prediction_improvement/without_rgb+only_rgb/26819+1/worst.pdf}
  \includegraphics[width=0.9\linewidth]{prediction_improvement/without_rgb+only_rgb/26819+1/best.pdf}
  \rule[1ex]{\textwidth}{.5pt}
  \includegraphics[width=0.9\linewidth]{prediction_improvement/without_rgb+only_rgb/9702+0/worst.pdf}
  \includegraphics[width=0.9\linewidth]{prediction_improvement/without_rgb+only_rgb/9702+0/best.pdf}
  \caption{%
    Best improvement from LiDAR to RGB.
  }%
  \label{fig:rgb-better-than-lidar}
\end{figure}
