This section will investigate the prediction of building outlines using data produced by the pipeline outlined in~\secref{sec:data}.
The U-Net model architecture presented in~\secref{sec:unet} is used for semantic segmentation.
We start by describing the general experimental setup in~\secref{sec:experimental-setup}.
A comparative investigation into the suitability of the different raster data types (aerial photography and/or LiDAR DSMs) for predicting building outlines is presented in~\secref{sec:features}.
In~\secref{sec:technique-experiments} we investigate if techniques intended to combat overfitting and increase training speed actually have their intended effect; specifically batch normalization, dropout, and data augmentation.
The LiDAR raster normalization methods presented in \algref{alg:local-min-max-scaling} and \algref{alg:metric-normalization} are implemented and compared in~\secref{sec:normalization-experiment}.
Finally, \secref{sec:loss-experiment} compares the empirical efficiency of different surrogate loss functions.
