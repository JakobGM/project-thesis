In order to create a ground truth segmentation mask, we must convert the vector-formatted mask polygons (buildings for instance) into the same rasterized format as the remote sensing data.
We construct discretized segmentation masks from vectorized mask polygons in the following manner:

\begin{itemize}
  \item Transform the mask polygons into the pixel coordinate system of the raster tile, using the affine transformation defined by the tile origin.
  \item Crop polygons outside the pixel region $(0, 255) \times (0, 255)$.
  \item Superimpose the cropped polygons on the discretized grid.
  \item Fill in the value $1$ for any pixel contained by the polygon exterior hulls, while not contained by any interior hull.
  \item Set remaining values to $0$.
\end{itemize}

A problem arises when pixels are partially contained by a polygon exterior and interior, i.e. when the pixel overlaps the polygon's boundary.
The pixel must be rather arbitrarily considered as either contained or not contained by the polygon.
Both decision rules are shown in \figref{fig:pixel-containment}.

\begin{figure}[H]
  \centering
  \begin{tikzpicture}[every node/.style={minimum size=1.3cm-\pgflinewidth, outer sep=0pt, fill=orange!80, fill opacity=0.4}, scale=1.3]
  \tikzmath{
    \gridheight=4;
    \gridwidth=4;
  }

  \def\cells{
    (0.5,0.5),
    (1.5,1.5),
    (2.5,2.5),
    (3.5,3.5),
    (3.5,2.5),
    (3.5,1.5),
    (3.5,0.5),
    (2.5,0.5),
    (1.5,0.5),
    (2.5,1.5),
    (2.5,3.5),
    (1.5,2.5)
  };
  \coordinate (offset) at (0.5, 0.5);
  \foreach \cell in \cells {
    \node at ($(\cell$) {};
  }

  \draw[step=1,color=black,draw opacity=0.6] (0,0) grid (\gridwidth,\gridheight);
  \draw[thick, orange, line width=0.1cm] (0.5, 0)
    -- (3.33, \gridheight)
    -- (\gridwidth, \gridheight)
    -- (\gridwidth, 0)
    -- cycle;
\end{tikzpicture}

  \hspace{2em}
  \begin{tikzpicture}[every node/.style={minimum size=1.3cm-\pgflinewidth, outer sep=0pt, fill=orange!80, fill opacity=0.4}, scale=1.3]
  \tikzmath{
    \gridheight=4;
    \gridwidth=4;
  }

  \def\cells{
    (3.5,2.5),
    (3.5,1.5),
    (3.5,0.5),
    (2.5,0.5),
    (2.5,1.5)
  };
  \coordinate (offset) at (0.5, 0.5);
  \foreach \cell in \cells {
    \node at ($(\cell$) {};
  }

  \draw[step=1,color=black,draw opacity=0.6] (0,0) grid (\gridwidth,\gridheight);
  \draw[thick, orange, line width=0.1cm] (0.5, 0)
    -- (3.33, \gridheight)
    -- (\gridwidth, \gridheight)
    -- (\gridwidth, 0)
    -- cycle;
\end{tikzpicture}

  \caption{
    The same polygon discretized to a raster grid using two different techniques.
    In the left figure, all pixels being \textit{touched} by the interior of the polygon
    are considered a part of the polygon, while in the left figure, only pixels
    entirely \textit{contained} within the interior are considered being part
    of the polygon.
  }
  \label{fig:pixel-containment}
\end{figure}
