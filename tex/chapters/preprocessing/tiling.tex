\begin{figure}
  \centering
  \begin{tikzpicture}[scale=0.07]
    \tikzmath{
      \xmin=0;
      \xmax=100;
      \ymin=0;
      \ymax=100;
      \xmid=0.5 * \xmin + 0.5 * \xmax;
      \ymid=0.5 * \ymin + 0.5 * \ymax;
    }
    \coordinate (lr) at (\xmax, \ymin);
    \coordinate (ul) at (\xmin, \ymax);

    \filldraw[cadastralcolor!40, draw=black] (50, \ymin) -- (62.5, 12.5) -- (25, 50) -- (\xmax, 50) -- (50, \ymax) -- (0, 50) -- cycle;
    \draw[dashed, color=cadastralcolor] (\xmin, \ymin) rectangle (\xmax, \ymax);

    \draw (ul) node[below right] {$(x_{\mathrm{min}}, y_{\mathrm{max}})$};
    \fill[cadastralcolor] (ul) circle [radius=1];

    \draw (lr) node[above left] {$(x_{\mathrm{max}}, y_{\mathrm{min}})$};
    \fill[cadastralcolor] (lr) circle [radius=1];

    \draw (\xmid, \ymax) node[above] {$w = x_{\mathrm{max}} - x_{\mathrm{min}}$};
    \draw (\xmax, \ymid) node[above, rotate=-90] {$h = y_{\mathrm{max}} - y_{\mathrm{min}}$};

    \draw (\xmid, 65) node[scale=1.5, black] {\textsf{CADASTRAL}};
  \end{tikzpicture}
  \caption{
    Bounding box calculation for a given cadastral.
    The cadastral is drawn in \textcolor{blue}{blue},
    and the resulting bounding box is drawn with \textcolor{blue}{blue} dashed lines.
  }
  \label{fig:cadastral_bbox}
\end{figure}

\begin{figure}
  \centering
  \begin{tikzpicture}[scale=0.05]
    \tikzmath{
      \tile=64;
      \bboxwidth=2.25 * \tile;
      \bboxheight=1.25 * \tile;
      \offset = 0.375 * \tile;
      \shift = 5;
    }
    \draw (0, 0) grid[step=\tile] (3 * \tile, 2 * \tile);
    \draw[dashed, color=cadastralcolor] (0 + \offset, 0 + \offset) rectangle (\bboxwidth + \offset, \bboxheight + \offset);
    \draw (-\shift, 0) edge[<->, dotted] node[above left] {$\SI{64}{\metre}~$} node[below left] {$\SI{256}{\pixel}$} (-\shift, \tile);
    \draw (0, -\shift) edge[<->, dotted] node[below, align=center] {$\SI{64}{\metre}$ \\ $\SI{256}{\pixel}$} (\tile, -\shift);
  \end{tikzpicture}
\end{figure}
