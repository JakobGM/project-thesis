Given a specific geographic region, defined by the extent of the cadastral plot, we must retrieve the raster which entirely covers the region of interest.
The simplest approach is to calculate the \textit{axis-aligned bounding box} of the plot, the minimum-area enclosing rectangle of the given plot.
A bounding box is uniquely defined by its centroid $\vec{c} = [\nicefrac{1}{2}(x_{\mathrm{min}} + x_{\mathrm{max}}), \nicefrac{1}{2}(y_{\mathrm{min}} + y_{\mathrm{max}})]$, width $w = x_{\mathrm{max}} - x_{\mathrm{min}}$, and height $h = y_{\mathrm{max}} - y_{\mathrm{min}}$, and denoted by $B(\vec{c}, w, h)$.
This is shown in \figref{fig:cadastral-bbox}.

\begin{figure}[htb]
  \captionsetup[subfigure]{position=b}
  \centering
  \subcaptionbox{
    Bounding box calculation for a given cadastral.
    The cadastral is drawn in \textcolor{orange}{orange},
    and the resulting bounding box is drawn with \textcolor{blue}{blue} dashed lines.
    \label{fig:cadastral-bbox}
  }{
    \begin{tikzpicture}[scale=0.07]
  \tikzmath{
    \xmin=0;
    \xmax=100;
    \ymin=0;
    \ymax=100;
    \xmid=0.5 * \xmin + 0.5 * \xmax;
    \ymid=0.5 * \ymin + 0.5 * \ymax;
  }
  \coordinate (lr) at (\xmax, \ymin);
  \coordinate (ul) at (\xmin, \ymax);

  \draw[dashed, color=cadastralcolor, fill=cadastralcolor!10] (\xmin, \ymin) rectangle (\xmax, \ymax);
  \filldraw[cadastralcolor!40, draw=black] (50, \ymin) -- (62.5, 12.5) -- (25, 50) -- (\xmax, 50) -- (50, \ymax) -- (0, 50) -- cycle;

  \draw (ul) node[above left] {$(x_{\mathrm{min}}, y_{\mathrm{max}})$};
  \fill[cadastralcolor] (ul) circle [radius=1];

  \draw (lr) node[below right] {$(x_{\mathrm{max}}, y_{\mathrm{min}})$};
  \fill[cadastralcolor] (lr) circle [radius=1];

  \draw (\xmid, \ymax) node[above] {$w = x_{\mathrm{max}} - x_{\mathrm{min}}$};
  \draw (\xmax, \ymid) node[above, rotate=-90] {$h = y_{\mathrm{max}} - y_{\mathrm{min}}$};

  \draw (\xmid, 65) node[scale=1.5, black] {\textsf{CADASTRAL}};
\end{tikzpicture}

  }
  \hspace{2em}
  \subcaptionbox{
    Figure showing the difference between a regular bounding box shown in
    \textcolor{blue}{blue}, and a minimum rotated rectangle shown in
    \textcolor{red}{red}.
    Angle of rectangle rotation denoted by $\phi$.
    \label{fig:rotated-bbox}
  }{
    \begin{tikzpicture}[scale=0.035]
  \tikzmath{
    \xmin=0;
    \xmax=100;
    \ymin=0;
    \ymax=100;
    \xmid=0.5 * \xmin + 0.5 * \xmax;
    \ymid=0.5 * \ymin + 0.5 * \ymax;
  }
  \coordinate (lr) at (\xmax, \ymin);
  \coordinate (ul) at (\xmin, \ymax);

  \draw[dashed, color=cadastralcolor, fill=cadastralcolor!20] (\xmin, \ymin) rectangle (\xmax, \ymax);
  \draw[dashed, color=red, fill=red!20, fill opacity=1] (\xmin, 50) -- (50, \ymin) -- (\xmax, 50) -- (50, \ymax) -- cycle; 
  \filldraw[cadastralcolor!40, draw=black] (50, \ymin) -- (62.5, 12.5) -- (25, 50) -- (\xmax, 50) -- (50, \ymax) -- (0, 50) -- cycle;
  \draw[dashed, thick, color=red] (\xmin, 50) -- (50, \ymin) -- (\xmax, 50) -- (50, \ymax) -- cycle; 

  \fill[cadastralcolor] (ul) circle [radius=1];
  \fill[cadastralcolor] (lr) circle [radius=1];

  \fill[red] (\xmin, 50) circle [radius=1];
  \fill[red] (\xmax, 50) circle [radius=1];
\end{tikzpicture}

  }
\end{figure}

The edges of the bounding box is by definition oriented parallel to the coordinate axes.
An alternative method is to calculate the \textit{arbitrarily oriented minimum bounding box} (AOMBB), a rectangle rotated by $\phi$ degrees w.r.t. the $x$-axis, as shown in \figref{fig:rotated-bbox}.

While AOMBB results in a region with less superfluous raster data, it requires warping of the original raw raster when $\phi \not\in \{ \SI{0}{\degree}, \SI{90}{\degree}, \SI{180}{\degree}, \SI{270}{\degree} \}$.
This requires data interpolation of the original raster data due to the rotation of the coordinate system, and may introduce artifacts in the warped raster data without careful parameter tuning.
AOMBB is therefore not a viable approach during the preprocessing stage, and we will therefore use axis-aligned minimum bounding boxes instead, from now on simply referred to as \textit{bounding boxes}.

Calculating bounding boxes for the cadastral plots in our data sets will yield rectangles of variable dimensions.
Variable input sizes will cause issues for model architectures which require predefined input dimensions.
Convolutional neural networks do handle variable input sizes, but dimensions off all images in a \textit{single} training batch must be of the same size.
It is therefore preferable to normalize the size of each bounding box.

The distributions of the bounding box widths ($w$), heights ($h$), and maximal dimensions ($m = \max \{w, h\}$) are shown in \figref{fig:bbox-stats}.

\begin{figure}[htb]
  \includegraphics[width=\linewidth]{bbox_stats}
  \caption{
    Distribution of bounding box widths $w$ (left), heights $h$ (middle), and largest dimension $m = \max \{w, h\}$ (right).
    The cut-off value of $\SI{64}{\meter}$ is shown by \textcolor{red}{red} dotted vertical lines.
    The fraction of bounding boxes with dimension $\leq \SI{64}{m}$ is annotated as well.
    The $x$-axis has been cut off at the 90th percentile.
    \textit{Dataset: Trondheim cadastre}.
  }
  \label{fig:bbox-stats}
\end{figure}

As can be seen in \figref{fig:bbox-stats}, the distributions of $h$ and $w$ are quite similar.
A square $1:1$ aspect ratio is therefore suitable for the normalized bounding box size.
Specifically, a $\SI{64}{\meter} \times \SI{64}{\meter}$ bounding box will be of sufficient size to contain $\approx \SI{85}{\percent}$ of all cadastre plots in a single tile.
With a LiDAR resolution of $\SI{0.2}{\meter}$, this results in a final image resolution of $\SI{256}{\pixel} \times \SI{256}{\pixel}$.
This is a common resolution for deep learning models, making comparison with earlier results simpler.

How should the bounding boxes be normalized to to $\SI{256}{\pixel} \times \SI{256}{\pixel}$?
A common technique is to resize the original image by use of methods such as bilinear interpolation or Lanczos resampling.
While this is tolerable for normal photographs, where each pixel has a variable real world area interpretation, it is an especially lossy transformation for remote sensing data.
In the Trondheim 2017 LiDAR data set, for instance, each pixel represents a $\SI{0.2}{\meter} \times \SI{0.2}{\meter}$ real world area.
If the (highly variable) extent of each bounding box is scaled to $\SI{256}{\pixel} \times \SI{256}{\pixel}$, the real world area of each pixel will differ greatly between cadastral plots.
Resized images will also become distorted when the original aspect ratio is not $1:1$.

A better method utilizes the fact that the remote sensing data covers a continuous geographic region, which allows us to expand the feature space beyond the original region of interest.
The original bounding box is denoted as $B(\vec{c}, w, h)$.
Now, define the following \enquote{enlarged} width and height:
%
\begin{align*}
  h^* &:= \ceil{\frac{h}{\SI{64}{\meter}}} \cdot \SI{64}{\meter},
  \hspace{3em}
  w^* := \ceil{\frac{w}{\SI{64}{\meter}}} \cdot \SI{64}{\meter}
\end{align*}
%
The new bounding box defined by $B(\vec{c}, w^*, h^*)$ covers the original bounding box and is divisible by \SI{256}{\pixel} in both dimensions.
With other words, the original bounding box is grown in all directions until both the width and height becomes a multiple of \SI{64}{\meter} and \SI{256}{\pixel}.
This is demonstrated in \figref{fig:bbox-growing}.

\begin{figure}[H]
  \centering
  \begin{tikzpicture}[scale=0.035]
  \tikzmath{
    \tile=64;
    \bboxwidth=2.25 * \tile;
    \collectionwidth=3 * \tile;
    \bboxheight=1.25 * \tile;
    \collectionheight=2 * \tile;
    \offset = 0.375 * \tile;
    \shift = 5;
  }
  \draw (0, 0) grid[step=\tile] (\collectionwidth, \collectionheight);

  % Show tile dimensions
  \draw[decoration={brace}, decorate]
    (-\shift, 0) -- node[above left] {$\SI{64}{\metre}~$} node[below left] {$\SI{256}{\pixel}$} (-\shift, \tile);

  \draw[decoration={brace,mirror}, decorate]
    (0, -\shift) -- node[below=2pt, align=center] {$\SI{64}{\metre}$ \\ $\SI{256}{\pixel}$} (\tile, -\shift);

  % Show dimensions of tile collection
  \draw[decoration={brace,mirror}, decorate]
    (\collectionwidth + \shift, 0) 
    --
    node[right=2pt]{$h^*$}
    (\collectionwidth + \shift, \collectionheight);
  \draw[decoration={brace}, decorate]
    (0, \collectionheight + \shift) 
    --
    node[above=2pt]{$w^*$}
    (\collectionwidth, \collectionheight + \shift);

  % Arrows showing growth of bounding box
  \tikzset{>=latex}
  \draw (\bboxwidth + \offset + 1, \offset - 1) edge[->, thick, dashed] (\collectionwidth - 1, 1);
  \draw (\bboxwidth + \offset + 1, \bboxheight + \offset + 1) edge[->, thick, dashed] (\collectionwidth - 1, \collectionheight - 1);
  \draw (\offset - 1, \bboxheight + \offset + 1) edge[->, thick, dashed] (1, \collectionheight);
  \draw (\offset - 1, \offset - 1) edge[->, thick, dashed] (1, 1);

  % Draw original bounding box
  \draw[dotted, thick, color=cadastralcolor, fill=cadastralcolor!10, fill opacity = 0.9] (0 + \offset, 0 + \offset) rectangle (\bboxwidth + \offset, \bboxheight + \offset);
\end{tikzpicture}

  \caption{
    Bounding box of width $2.25 \times \SI{64}{\meter} = \SI{144}{\meter}$ and height $1.25 \times \SI{64}{\meter} = \SI{80}{\meter}$.
    The bounding box is grown until it is 3 tiles wide and 2 tiles tall, i.e. $\SI{192}{\meter} \times \SI{128}{\meter}$.
  }
  \label{fig:bbox-growing}
\end{figure}

The resulting bounding box can now be divided into $\nicefrac{w^*h^*}{64^2}$ tiled images of resolution $\SI{256}{\pixel} \times \SI{256}{\pixel}$, all of which with a real world pixel area of $\SI{0.2}{\meter} \times \SI{0.2}{\meter}$, and no spatial information has been lost.
Each tile's geographic extent is uniquely defined by the coordinate of the upper left corner (\textit{tile origin}), since the tile dimensions are identical.
An affine transformation from the given UTM zone into the tile's discretized coordinate system can be constructed from the same coordinate.

The additional area $B(\vec{c}, w, h) \setminus B(\vec{c}, w^*, h^*)$ is filled with real raster data and respective target masks, and therefore may cause expanded cadastre bounding boxes to partially overlap.
This will result in certain cadastral plots to share features, and must therefore be carefully dealt with in order to prevent data leakage across training, validation, and test splits.

Another approach is to fill in the additional area with zero values, effectively preventing all data leakage between cadastral plots.
A disadvantage with this approach is that all models are now required to learn to ignore this additional, fake data, and this could possibly result in reduced predictive performance and/or longer training times.
