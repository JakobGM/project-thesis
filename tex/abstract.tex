\begin{abstract}
  We present a system for predicting building footprints by using remote sensing data in the form of aerial photography (RGB) and LiDAR elevation measurements (DSMs).
  Techniques for converting data from conventional GIS vector and raster formats to a format suitable for machine learning purposes are discussed.
  The U-Net model architecture is used in order to train several model variants on a labeled building footprint dataset covering the Norwegian municipality of Trondheim.
  A model using aerial photography in combination with LiDAR elevation data achieved the best mean IoU test score.
\end{abstract}
