% Add support for UTF-8 character set for those engines that do not support it natively.
\usepackage[utf8]{inputenc}

% Very common math environments such as \begin{align}, and much, much more.
% Docs: http://texdoc.net/texmf-dist/doc/latex/amsmath/amsldoc.pdf
\usepackage{amsmath}

% Extended symbol collection
\usepackage{amssymb}

% Enhanced version of \newtheorem{}, which also allows non-numbered theorems.
\usepackage{amsthm}

% Add \od{}{} for ordinary derivatives, \pd{}{} for partial derivatives, and \del{} for delimiters.
% Docs: http://ctan.math.washington.edu/tex-archive/macros/latex/contrib/commath/commath.pdf
\usepackage{commath}

% Allows definition of colors using \definecolor{}{}{}
\usepackage{xcolor}

% For including source code with syntax highlighting.
% Docs: https://github.com/gpoore/minted/blob/master/source/minted.pdf
\usepackage{minted}
% Enables the use of \begin{pythoncode}
\newminted{python}{python3=true}
% Enables the use of \begin{shellcode}
\definecolor{bgshell}{rgb}{0.95,0.95,0.95}
\newminted{shell}{bgcolor=bgshell,fontfamily=tt}

% Sitation management.
% Docs: http://ctan.uib.no/macros/latex/contrib/biblatex/doc/biblatex.pdf
% Tutorial: https://www.overleaf.com/learn/latex/Articles/Getting_started_with_BibLaTeX
\usepackage[style=numeric,backend=biber]{biblatex}
% Specify file containing all bibliographic metadata.
\addbibresource{bibliography.bib}

% Create figures with semantic code
\usepackage{tikz}
% Allow calculations within curly brackets in tikz figures
\usetikzlibrary{calc, math}
% Adds the braces decoration for showing distances in tikz figures
\usetikzlibrary{decorations.pathreplacing}
% For drawing matrices, useful for masking demonstration
\usetikzlibrary{matrix}

% Proper formatting of units by using \SI{value}{\unit}
\usepackage{siunitx}
\DeclareSIUnit\pixel{px}

% For some reason, this package prevents some side-by-side tikz figures from wrapping
\usepackage[a4paper]{geometry}

% For insterting images into figures
\usepackage{graphicx}
\graphicspath{ {./img/} }

% Better formatting of URLs
\usepackage{url}

% Used for sectionsummary environment
\usepackage{tcolorbox}
